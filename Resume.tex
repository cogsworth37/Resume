%%%%%%%%%%%%%%%%%%%%%%%%%%%%%%%%%%%%%%%%%
% Developer CV
% LaTeX Template
% Version 1.0 (28/1/19)
%
% This template originates from:
% http://www.LaTeXTemplates.com
%
% Authors:
% Jan Vorisek (jan@vorisek.me)
% Based on a template by Jan Küster (info@jankuester.com)
% Modified for LaTeX Templates by Vel (vel@LaTeXTemplates.com)
%
% License:
% The MIT License (see included LICENSE file)
%
%%%%%%%%%%%%%%%%%%%%%%%%%%%%%%%%%%%%%%%%%

%----------------------------------------------------------------------------------------
%	PACKAGES AND OTHER DOCUMENT CONFIGURATIONS
%----------------------------------------------------------------------------------------

\documentclass[9pt]{developercv} % Default font size, values from 8-12pt are recommended

%----------------------------------------------------------------------------------------

\definecolor{mygrey}{HTML}{bdbdbd}

\begin{document}

%----------------------------------------------------------------------------------------
%	TITLE AND CONTACT INFORMATION
%----------------------------------------------------------------------------------------

\begin{minipage}[t]{0.45\textwidth} % 45% of the page width for name
	\vspace{-\baselineskip} % Required for vertically aligning minipages
	
	% If your name is very short, use just one of the lines below
	% If your name is very long, reduce the font size or make the minipage wider and reduce the others proportionately
	\colorbox{white}{{\HUGE\textcolor{black}{\textbf{\MakeUppercase{Andy}}}}} % First name
	
	\colorbox{white}{{\HUGE\textcolor{black}{\textbf{\MakeUppercase{Quangvan}}}}} % Last name
	
	\vspace{6pt}
	
	{\huge DevOps Practitioner} % Career or current job title
\end{minipage}
\begin{minipage}[t]{0.45\textwidth} % 27.5% of the page width for the first row of icons
	\vspace{-\baselineskip} % Required for vertically aligning minipages
	
	% The first parameter is the FontAwesome icon name, the second is the box size and the third is the text
	% Other icons can be found by referring to fontawesome.pdf (supplied with the template) and using the word after \fa in the command for the icon you want
	\icon{MapMarker}{12}{Smithville, MO}\\
	\icon{Phone}{12}{(319) 541-6905}\\
	\icon{EnvelopeO}{12}{\href{mailto:andy.quangvan@gmail.com}{andy.quangvan@gmail.com}}\\	
\end{minipage}
%\begin{minipage}[t]{0.275\textwidth} % 27.5% of the page width for the second row of icons
%	\vspace{-\baselineskip} % Required for vertically aligning minipages
	
	% The first parameter is the FontAwesome icon name, the second is the box size and the third is the text
	% Other icons can be found by referring to fontawesome.pdf (supplied with the template) and using the word after \fa in the command for the icon you want
	%\icon{Globe}{12}{\href{https://alyx.vance.me}{alyx.vance.me}}\\
	%\icon{Github}{12}{\href{https://github.com/alyxvance}{github.com/alyxvance}}\\
	%\icon{Twitter}{12}{\href{https://twitter.com/@alyxvance}{@alyxvance}}\\
%\end{minipage}

\vspace{0.5cm}

%----------------------------------------------------------------------------------------
%	INTRODUCTION, SKILLS AND TECHNOLOGIES
%----------------------------------------------------------------------------------------

\cvsect{About Me - Skills}

\begin{minipage}[t]{0.43\textwidth} % 40% of the page width for the introduction text
	\vspace{-\baselineskip} % Required for vertically aligning minipages
	
	I am a problem solver and a change agent. 
	I improve processes to those that make sense instead of settling for doing things the way it's always been done. I am an avid learner and a people enabler.
\end{minipage}
\hfill % Whitespace between
\begin{minipage}[t]{0.5\textwidth} % 50% of the page for the skills bar chart
	\vspace{-\baselineskip} % Required for vertically aligning minipages
	\begin{barchart}{5.5}
		\baritem{JavaScript}{20}
		\baritem{Python}{75}
		\baritem{Go}{70}
		\baritem{Azure}{90}
		\baritem{AWS}{90}
		\baritem{DevOps}{85}
		\baritem{SRE}{85}
		\baritem{SecOps}{60}
		\baritem{CSharp}{40}
	\end{barchart}
\end{minipage}

%----------------------------------------------------------------------------------------
%	EXPERIENCE
%----------------------------------------------------------------------------------------

\cvsect{Experience}

\begin{entrylist}
	\entry
		{2021 -- present}
		{Lead Site Reliability Engineer}
		{H\&R Block}
		{Created an observability platform that helped teams acheive reliability \\
		Helped teams refactor code to increase observability for workflows \\
		Introduced more robust health checks that didn't simply "return 200" \\
		Created dashboards and workbooks that allowed dev teams to see the reliability of their applications \\
		Worked on dockerizing applications and introducing Kubernetes \\
		Architected a way for teams to stabilize their Monitoring platform (Grafana/Prometheus)\\
		Created a standardized practice for reliability in Eventing Platforms \\
		Implemented unit testing by writing and training on Unit Test Principals \\
		Established scaling profiles to cut costs and improve reliability \\
		Migrated teams from on-prem to the cloud \\
		\texttt{Azure}\slashsep\texttt{DevOps}\slashsep\texttt{SRE}\slashsep\texttt{Kubernetes}\slashsep\texttt{Docker}\slashsep\texttt{CSharp}}
	\entry
		{2021 -- 2021}
		{Information Security Cloud Architect}
		{Garmin}
		{Helped Garmin create a strategy that would secure their public cloud presence\\ 
		Create a strategy to onboard new acquisitions \\
		Helped choose and implement their CSPM \\
		Helped create an IAM process for the cloud to start utilizing RBAC \\
		\texttt{Azure}\slashsep\texttt{Architecture}\slashsep\texttt{CSPM}}
	\entry
		{2019 -- 2021}
		{IT Operations Manager}
		{H\&R Block}
		{Introduced new technologies to development teams e.g. Scale Sets, configuration management, Packer \\
		Created processes in which new technologies should be adapted by highlighting Architectural patterns and common pitfalls \\
		Worked with Engineering teams to implement new technologies \\
		Created platforms that the engineers could interact with through their pipelines \\
		Worked with developers to help them think about their apps in Cloud-Centric ways e.g. Database connection pooling and retry logic \\
		Worked with operations to think about managing infrastructure in Cloud-Centric ways. e.g. Terraform, Chef, Ansible \\
		Automated manual procedures to move faster e.g.Infrastructure as code, Adding users to groups \\
		Helped define culture to move the company forward faster e.g. Fail fast, cloud first, agile, blameless post mortems \\
		Increased observability by implementing monitoring as a platform\\
		Increased responsiveness to the incident management process\\
		Created automation to help facilitate business processes\\ 
		\texttt{Azure}\slashsep\texttt{DevOps}\slashsep\texttt{SRE}\slashsep\texttt{Leadership}}
	\entry
		{2017 -- 2019}
		{Senior Engineering Manager}
		{Rx Savings Solutions}
		{Managed entire AWS Infrastructure \\
		Created a CI/CD Pipeline \\
		Lead a team of System Engineers \\
		Converted System Engineers to Site Reliability Engineers \\
		Created security policies to help increase Cyber Defense \\
		Implemented Cyber Security applications including DLP and EDR \\
		Created Change Management policies \\
		Implemented Docker Containers\\ 
		\texttt{Docker}\slashsep\texttt{SRE}\slashsep\texttt{Leadership}\slashsep\texttt{AWS}}
	\entry
		{2014 -- 2017}
		{System Architect}
		{Cerner}
		{Automated workflows to be more efficient \\
		 Wrote ruby scripts to interact against REST api’s \\
		 Wrote Chef cookbooks to automate deployments \\
		Implemented Rundeck to automate deployment steps \\
		Managed a Big Data cluster using Cloudera Hadoop \\
		Managed the cloud infrastructure \\
		Worked with our Development team to improve operations \\
		Wrote scripts to measure metrics to better monitor systems and applications \\
		\texttt{Linux}\slashsep\texttt{Chef}\slashsep\texttt{Ruby}}
\end{entrylist}
%----------------------------------------------------------------------------------------
%	EDUCATION
%----------------------------------------------------------------------------------------

\cvsect{Education}

\begin{entrylist}
	\entry
		{August 2019}
		{Master's Degree, Information Technology}
		{Southern New Hampsire University}
		{}
	\entry
		{October 2011}
		{Bachelor's Degree, Computer Network Management}
		{Westwood College}
		{}
\end{entrylist}

\end{document}
